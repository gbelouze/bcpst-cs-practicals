\correctionExam{Intégration numérique}{27-05-2023}

\section*{Méthode de Simpson}

\ques On regarde $ f: x \mapsto x^4 $, $ a=-1 $, $ b=1 $. On a alors 
\begin{align*}
    I_{a, b}(f) &= \frac{2}{5}\\
    I^{(\textrm{Simpson})}_{a, b}(f) &= \frac{2}{3}\\
                                     &\neq I_{a, b}(f)
\end{align*}

\ques Soit $ f : x \mapsto ux^2 + vx + w $. On calcule 
\begin{align*}
    I^{(\textrm{Simpson})}_{a, b}(f) &= \frac{b-a}{6} \left( u(b^2 + a^2 + (b+a)^2) + v(b + a + 2(b+a)) + 6w \right)\\
    I_{a, b}(f) &= u \frac{b^3 - a^3}{3} + v \frac{b^2-a^2}{2} + w(b-a)\\
                &= (b-a) \left( \frac{u}{3}(b^2 - ab + a^2) + \frac{v}{2}(b+a) + w \right)\\
                &= \frac{b-a}{6} \left( u(b^2 + b^2 - 2ab + a^2 + a^2) + v(3b+3a) + 6w \right)\\
                &= I^{(\textrm{Simpson})}_{a, b}(f)
\end{align*}
ce qui conclut.

\ques On calcule à nouveau 
\begin{align*}
    I^{(\textrm{Simpson})}_{a, b}(P_{a,b}) &= \frac{b-a}{6}((\frac{a+b}{2})^3 + (-\frac{a+b}{2})^3) \\
                                           &= 0\\
    I_{a, b}(f) &= \int_{a}^{b} (t - \frac{a+b}{2})dt\\
                &= \int_{-\frac{a+b}{2}}^{\frac{a+b}{2}} t^3dt \\
                &= 0\\
\end{align*}

\ques On remarque que la méthode de Simpson est linéaire, i.e. \[
    I^{(\textrm{Simpson})}_{a, b}(\alpha f + g) = \alpha  I^{(\textrm{Simpson})}_{a, b}(f) +  I^{(\textrm{Simpson})}_{a, b}(g)
\]
Ainsi soit $ P \in \R_3[X] $, $ a, b \in \R $. On écrit $ P = \alpha P_{a, b} + Q $ grâce à la question précédente. On a alors 
\begin{align*}
    I^{(\textrm{Simpson})}_{a, b}(P) &= \alpha  I^{(\textrm{Simpson})}_{a, b}(P_{a,b}) + I^{(\textrm{Simpson})}_{a, b}(Q) \\
                                     &= \alpha I_{a, b}(P_{a, b}) + I^{(\textrm{Simpson})}_{a, b}(Q) \quad \textrm{grâce à la question 3.a}\\
                                     &= \alpha I_{a, b}(P_{a, b}) + I_{a, b}(Q) \quad \textrm{grâce à la question 2}\\
                                     &= I_{a, b}(P) \quad \textrm{par linéarité de l'intégrale}\\
\end{align*}
ce qui conclut.

\quessques \begin{minted}{python}
    def simpson(f, a, b):
    return (b-a)/2 * (f(a) + f(b) + f((a+b)/2))
\end{minted}

\ssques \begin{minted}{python}
    def simpson_composite(f, a, b, n):
        x = np.linspace(a, b, n)
        tot = 0
        for k in range(n-1):
            tot += simpson(f, x[k], x[k+1])
        return tot
\end{minted}

\ques La méthode des rectangles est d'ordre $ 0 $. Il faut lui préférer la méthode de Simpson: l'erreur de la généralisation composite de la méthode des rectangles décroit en $ O(\frac{1}{n}) $ tandis que celle de la généralisation composite de la méthode de Simpson décroit en $ O(\frac{1}{n^4}) $. Remarquez qu'on ne demande pas de se souvenir de la puissance de décroissance exacte, il est suffisant de mentionner qu'expérimentalement, l'erreur de Simpson décroit plus rapidement avec $ n $ que celle des rectangles.


\section*{Gymnastique Python}

\quessques \begin{minted}{python}
    def trouve_minimum(L):
        mini = L[0]
        for x in L:
            if x < mini:
                mini = x
        return mini
\end{minted}

\ssques \begin{minted}{python}
    def trouve_indice_du_minimum(L):
        mini, argmini = L[0], 0
        for i, x in enumerate(L):
            if x < mini:
                mini = x
                argmini = i
        return argmini
\end{minted}

Si on ne connait pas la fonction \mintinline{python}{enumerate}, on peut également écrire
\begin{minted}{python}
    def trouve_indice_du_minimum(L):
        mini, argmini = L[0], 0
        for i in range(len(L)):
            x = L[i]
            if x < mini:
                mini = x
                argmini = i
        return argmini
\end{minted}

\ques 

\begin{minted}{python}
    def sous_liste_pair(L):
        ret = []
        for x in L:
            if x%2 == 0:
                ret.append(x)
        return ret
\end{minted}

\ques 

\begin{minted}{python}
    def applati(L):
        ret = []
        for sous_liste in L:
            for x in sous_liste:
                ret.append(x)
        return ret
\end{minted}
ou encore 

\begin{minted}{python}
    def applati(L):
        ret = []
        for x in L:
            ret += x
        return ret
\end{minted}

\ques \mintinline{python}{L.append(a)}

\quessques \mintinline{python}{True}
\ssques \mintinline{python}{True}
\ssques \mintinline{python}{False}

\quessques \mintinline{python}{[3]}
\ssques \mintinline{python}{"cpst"}
\ssques \begin{minted}{python}
    2
    4
\end{minted}
