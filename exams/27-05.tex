\exam{Intégration numérique}{27-05-2023}

Les barèmes sont indicatifs et pourront être reconsidérés.

\section*{Méthode de Simpson (24 points)}

Soit $ f : \R \rightarrow \R $ une fonction continue. La méthode de Simpson est une méthode pour numériquement calculer $ I^{(\textrm{Simpson})}_{a, b}(f) $ une approximation de $ \int_{a}^{b} f(t)dt $, avec la formule \[
    I^{(\textrm{Simpson})}_{a, b}(f) = \frac{b-a}{6} (f(a) + f(b) + 4 f(\frac{a+b}{2}))
\]

On rappelle qu'une méthode d'intégration numérique est d'ordre $ k $ si elle fournit un résultat exact lorsque $ f $ est un polynôme de degré inférieur ou égal à $ k $.

\ques (3 points) Montrer que la méthode de Simpson n'est pas d'ordre $ 4 $.
\ques (5 points) Montrer que la méthode de Simpson est d'ordre au moins $ 2 $.
\quessques (2 points) Montrer que la méthode de Simpson donne un résultat exact pour la fonction $ P_{a,b} : x \mapsto (x - \frac{a+b}{2})^3 $.
\ssques  (3 points) Soit $ P \in \P_3[X] $ un polynôme de degré au plus $ 3 $. Montrer qu'il existe $ \alpha \in \R, Q \in \P_2[X] $ tel que \[
    P = \alpha P_{a, b} + Q
\]
\ques (5 points) En utilisant la linéarité de l'intégrale, déduire des questions précédentes que la méthode de Simpson est d'ordre $ 3 $.

\quessques (3 points) Implémenter une fonction \mintinline{python}{simpson(f, a, b)} qui calcule l'approximation de $ \int_{a}^{b} f $ donnée par la méthode de Simpson.
\ssques (5 points) En déduire une fonction \mintinline{python}{simpson_composite(f, a, b, n)} qui calcule la généralisation composite de la méthode de Simpson, c'est-à-dire qui calcule \[
    I^{(\textrm{Simpson})}_{a, b, n}(f) = \sum_{k=0}^{n-1} I^{(Simpson)}_{x_k, x_{k+1}}(f)
\]
où $ (x_0, \ldots, x_n) $ forme une subdivision régulièrement espacée de l'intervalle $ [a, b] $. 

On rappelle que la fonction \mintinline{python}{np.linspace(a, b, n)} fournit une telle subdivision.

\ques (3 points) Rappeler l'ordre de la méthode des rectangles. En général, faut-il préférer la méthode des rectangles ou la méthode de Simpson ? Expliquer la différence entre les deux méthodes que l'on observe expérimentalement qui justifie cette préférence.


\section*{Gymnastique Python (18 points)}

Les exercices de cette section sont indépendants, et testent votre maîtrise générale de Python.

\quessques (3 points) Écrire une fonction \mintinline{python}{trouve_minimum(L)} qui prend en argument une liste non vide d'entiers \texttt{L} et renvoie le plus petit élément de \texttt{L}. On se refusera d'utiliser la fonction Python \mintinline{python}{min}.
\ssques (3 points) Modifier la fonction pour obtenir \mintinline{python}{trouve_indice_du_minimum(L)} que renvoie \textit{l'indice} du plus petit élément de \texttt{l}. S'il y a plusieurs indices où \texttt{L} atteint son minimum, vous êtes libres de choisir quel indice est renvoyé.

\quessques (3 points) Écrire une fonction \mintinline{python}{zeros(n)} qui prend en argument un entier \texttt{n} et renvoie une liste à \texttt{n} coefficients, tous nuls. Ainsi, \mintinline{python}{zeros(3)} renvoie la liste \mintinline{python}{[0, 0, 0]}.
\ssques (3 points) En déduire une fonction \mintinline{python}{zeros2(n, p)} qui prend en arguments deux entiers \texttt{n} et \texttt{p} et renvoie une liste de \texttt{n} listes, contenant chacune \texttt{p} coefficients, tous nuls. Ainsi, \mintinline{python}{zeros2(3, 4)} devra renvoyer la liste \mintinline{python}{[[0, 0, 0, 0], [0, 0, 0, 0], [0, 0, 0, 0]]}.

\ques (1 point) À partir d'une liste \texttt{L} et d'un élément \texttt{a}, comment obtenir la liste où l'on a ajouté \texttt{a} à la fin de \texttt{L}.

\quessques (1 point) Que vaut \mintinline{python}{True or False} ?
\ssques (1 point) Que vaut \mintinline{python}{False or False and True} ?
\ssques (1 point) Que vaut \mintinline{python}{(False or False) and True} ?

\quessques (1 point) Qu'affiche le programme suivant ?

\begin{minted}{python}
    L = [1, 3, 5, 7]
    print(L[1:2])
\end{minted}

\ssques (1 point) Qu'affiche le programme suivant ? 

\begin{minted}{python}
    s = "bcpst"
    print(s[1:])
\end{minted}

\ssques (1 point) Qu'affiche le programme suivant ? 

\begin{minted}{python}
    for i in range(2, 6, 2):
        print(i)
\end{minted}
