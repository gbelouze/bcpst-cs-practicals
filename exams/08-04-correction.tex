\correctionExam{Manipulation d'images, requêtes SQL}{18-03-2023}

\section*{Requêtes SQL}
\ques Les espaces, l'indentation, les retours à la ligne, le choix de majuscule/minuscule n'importent pas.
\ssques \mintinline{sql}{SELECT Title FROM publication;}
\ssques \begin{minted}{sql}
    SELECT Title, Year
        FROM publication
        ORDER BY Year ASC
        LIMIT 5;
\end{minted}
\ssques \begin{minted}{sql}
    SELECT Count(*)
        FROM publication
        WHERE Year > 1980;
\end{minted}

\ssques \begin{minted}{sql}
    SELECT p.Title
        FROM publication as p JOIN author as a ON a.Article = p.DOI
        WHERE a.Name = "Curie" AND a.FirstName = "Marie";
\end{minted}

\ssques \begin{minted}{sql}
    SELECT a.Name, p.Title
        FROM publication as p JOIN author as a ON a.Article = p.DOI
        WHERE a.Rank = 1 AND p.Year > 2000 AND p.Field = "Biology";
\end{minted}

\ssques \begin{minted}{sql}
    SELECT a1.Name
        FROM author as a1 JOIN author as a2 ON a1.DOI = a2.DOI
        WHERE (a1.Name != "Turing" OR a1.FirstName != "Alan")
            AND a2.Name = "Turing" AND a2.FirstName = "Alan";
\end{minted}

\section*{Manipulation d'images}

\ques
\ssques $ M^{\leftrightarrow} $ a la même dimension que $ M $, donc $ m \times n $
\ssques $ M^{\leftrightarrow}_{i,j} = M_{i, n-j} $
\ssques \begin{minted}{python}
    def  reflexion_horizontale(img):
        h, w = np.shape(img)
        ret = np.zeros((h, w))
        for i in range(h):
            for j in range(w):
                ret[i, j] = img[i, w-1-j]
        return ret

    # ou encore
    def  reflexion_horizontale(img):
        return img[:, ::-1]
\end{minted}

\ques
\ssques $ M^T $ a dimension $ n \times m $.
\ssques $ M^t_{i,j} = M_{j, i} $
\ssques \begin{minted}{python}
    def transpose(img):
        h, w = np.shape(img)
        ret = np.zeros((w, h))
        for i in range(w):
            for j in range(h):
                ret[i, j] = img[j, i]
    return ret

    # ou encore
    def transpose(img):
        return img.T
\end{minted}

\ques
\ssques $ M^{\updownarrow} $ a la même dimension que $ M $, donc $ m \times n $
\ssques $ M^{\updownarrow}_{i, j} = M_{m-i, j} $
\ssques On montre l'égalité coefficient par coefficient :
\begin{align*}
    (M^T)^{\leftrightarrow}_{i,j} &= M^T_{i, m-j} \quad \textrm{car $ M^T $ est de dimension $ n \times m $} \\
                                  &= M_{m-j, i} \\
    (M^{\updownarrow})^T_{i,j} &=  M^{\updownarrow}_{j,i} \\
                               &= M_{m-j, i}
\end{align*}
\ssques Il suffit de prendre la transposée de l'égalité prouvée à la question précédente.

\ques 

\begin{minted}{python}
    def reflexion_verticale(img):
        return transpose(reflexion_horizontale(transpose(img)))

    # pas la solution attendue, mais on peut aussi écrire
    def reflexion_verticale(img):
        return img[::-1, :]
\end{minted}

