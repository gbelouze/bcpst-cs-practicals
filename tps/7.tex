%! TEX root = ../main.tex

\titre{Dictionnaires}

%\commentaire{\textit{Un nouveau type de variable : les dictionnaires}\\
Ce sont des structures de données (\texttt{dict}) modifiables et non séquentielles ; un élément n'est pas repéré à l'aide d'un indice mais à l'aide d'un nom (sa \texttt{clef}). Un élément est constitué d'un \texttt{champ} : la donnée d'une clé et d'une valeur.
\begin{enumerate}
	\item La clé est un nombre, une chaîne de caractère (ou un t-uplet, mais ce n'est pas un type de variable qu'on a vu jusque là).
	\item La valeur peut être de type quelconque.
\end{enumerate}
Le nom de dictionnaire ne vient pas de nulle part : cela fonctionne comme un dictionnaire papier, où on range une défintion (un champs) en la repérant grâce à un certain mot (la clef).\\
Le type \texttt{dict} permet la définition de conteneurs dont les valeurs sont repérées non plus par des indices mais par des clés. \\

Visuellement, un dictionnaire est délimité par des acolades : \texttt{\{\}}, et les clefs sont reliées aux champs grâce à deux points : \texttt{:}.\\

Comment définir un dictionnaire ?\\
Un exemple :
\begin{minted}{python}
dict_test = {'clef1':12.5, 'clef2':'deuxieme champs', 3:'champs3'}
\end{minted}
On peut aussi le faire en plusieurs etapes :
\begin{minted}{python}
dict_test2 = {}
dict_test2['clef1'] = 12.5
dict_test2['clef2'] = 'deuxieme champs'
dict_test2[3] = 'champs3'
\end{minted}

On peut ensuite accéder aux champs contenus dans un dictionnaire en appelant les clefs : \mintinline{python}{dict_test['clef1']} renvoie 12.5.
%}

\exo
\textit{Entrainement}
\quessques Créez un dictionnaire \texttt{age} contenant les âges de trois personnes différentes, en les repérant par leur prénom, puis un dictionnaire \texttt{taille} contenant leur taille.
\ssques Les valeurs (champs) d'un dictionnaire peuvent aussi être des dictionnaires ! Faites un dictionnaire \texttt{informations} contenant les deux dictionnaires précédents, repérés par une chaîne de caractère chacun.
\ssques Faites un nouveau dictionnaire \texttt{ville\_naissance} contenant, pour les mêmes personnes, leur ville de naissance. Ajoutez ce dictionnaire à \texttt{informations}.

\ques \textit{Opérations sur les dictionnaires}
\ssques Que renvoie la commande suivante ?
\begin{minted}{python}
for x in age :
    print(x)
\end{minted}
Que comprenez-vous sur cette manière de parcourir un dictionnaire ?
\ssques Parcourez chaque dictionnaire contenu dans \texttt{informations} (à l'aide d'une boucle bien sûr), et enlevez une personne des dictionnaires que vous avez faits jusque-là. La fonction \texttt{del} permet d'enlever une clef (et le champ associé) dans un dictionnaire.

\newpage
\ques \textit{Méthodes applicables aux dictionnaires}
Testez les méthodes suivantes sur les dictionnaires dont vous disposez : \\
- \mintinline{python}{D.keys()} (pour un dictionnaire \mintinline{python}{D})\\
- \mintinline{python}{D.values()} \\
- \mintinline{python}{D.items()} \\
- \mintinline{python}{D.copy()} \\
Que comprenez-vous sur ces méthodes, que font-elles ?

\exo
\textit{Cryptage de César à l'aide de dictionnaires}
\ques  Définir une variable \mintinline{python}{alphabet = "abcdefghijklmnopqrstuvwxyz"}.
\ques Ecrire un script qui définit un dictionnaire ayant pour clefs les lettre de l'alphabet et pour valeurs leur position dans l'alphabet. Utilisez bien sûr la variable \texttt{alphabet} !
\ques Ecrire une fonction \texttt{cesar(n)} prenant en argument un entier \texttt{n} et qui renvoie un dictionnaire comme en 2, mais dont les valeurs sont décalées de l'entier \texttt{n} modulo 26.
\ques Ecrire une fonction \texttt{cryptage(texte,n)} prenant en argument un texte sous forme d'une chaine de caractère et un entier \texttt{n} et qui le crypte selon le chiffrement de César avec décalage de \texttt{n}.
