\section{Aide-mémoire}

\begin{table}[h!]
	\centering
	\begin{tabular}{|ll|}
		\hline
		\mintinline{sql}{SELECT}                       & sélection des colonnes                              \\
		\mintinline{sql}{SELECT DISTINCT}              & sélection sans doublon                              \\
		\mintinline{sql}{FROM table}                   & nom d'une table                                     \\
		\mintinline{sql}{WHERE condition}              & imposer une condition                               \\
		\mintinline{sql}{GROUP BY expression}          & grouper les résultats                               \\
		\mintinline{sql}{HAVING condition}             & condition sur un groupe                             \\
		\mintinline{sql}{ORDER BY expression ASC/DESC} & trier les résultats par ordre croissant décroissant \\
		                                               &                                                     \\
		\mintinline{sql}{LIMIT n}                      & limiter à $ n $ enregistrements                     \\
		\mintinline{sql}{OFFSET n}                     & débuter à partir de $ n $ enregistrements           \\
		                                               &                                                     \\
		\mintinline{sql}{UNION | INTERSECT | EXCEPT}   & Opérations ensemblistes sur les requêtes            \\
		\hline
	\end{tabular}
	\caption{Syntaxe de base}
\end{table}

\begin{table}[h!]
	\centering
	\begin{tabular}{|ll|}
		\hline
		\mintinline{sql}{COUNT()} & nombre d'enregistrements      \\
		\mintinline{sql}{MAX()}   & valeur maximale d'un attribut \\
		\mintinline{sql}{MIN()}   & valeur minimale d'un attribut \\
		\mintinline{sql}{SUM()}   & somme d'un attribut           \\
		\mintinline{sql}{AVG()}   & moyenne d'un attribut         \\
		\hline
	\end{tabular}
	\caption{Fonctions d'agrégation}
	\label{tab:agregation-func}
\end{table}


Structure d'un \mintinline{sql}{JOIN}:

\begin{minted}{sql}
    SELECT t1.name1, t2.name2
        FROM table1 AS t1
            JOIN table2 As t2
            ON t1.col1 = t2.col2
        WHERE condition
\end{minted}

Ou même d'un \mintinline{sql}{JOIN} multiple
\begin{minted}{sql}
    SELECT t1.name1, t2.name2, t3.name3
        FROM table1 AS t1
            JOIN table2 AS t2 ON t1.col1 = t2.col2
            JOIN table3 AS t3 ON ...
            JOIN table4 AS t4 ON ...
        WHERE condition
\end{minted}

Structure d'une agrégation:

\begin{minted}{sql}
    SELECT name
        FROM table
        GROUP BY column
        HAVING condition
\end{minted}

Lors d'une agrégation, on peut faire intervenir les fonctions d'agrégations \ref{tab:agregation-func} pour réaliser des calculs au sein de chaque groupe.


\section{Exercices}

On dispose des tables décrites \autoref{fig:mondiale}. On marque par $ \star $ les colonnes qui forment la clef primaire de la table.
\begin{figure}[h!]
	\centering
	\subfloat[Table \texttt{country}]{
		\begin{tabular}{|cccccc|}
			\hline
			\texttt{Name} & \texttt{Code}$ \star $ & \texttt{Capital} & \texttt{Province} & \texttt{Area}                        & \texttt{Population} \\
			              &                        &                  &                   & Surface du pays en $ \textrm{km}^2 $ &                     \\
			\hline
		\end{tabular}
	}%
	\qquad
	\subfloat[Table \texttt{encompasses}]{
		\begin{tabular}{|ccc|}
			\hline
			\texttt{Country}$ \star $                 & \texttt{Continent}$ \star $ & \texttt{Percentage}                                 \\
			Clef étrangère pour \texttt{country.Code} &                             & Pourcentage de la surface du pays dans le continent \\
			\hline
		\end{tabular}
	}%
	\qquad
	\subfloat[Table \texttt{city}]{
		\begin{tabular}{|cccccc|}
			\hline
			\texttt{Name}$ \star $ & \texttt{Country}$ \star $                 & \texttt{Province} & \texttt{Population} & \texttt{Longitude} & \texttt{Latitude} \\
			                       & Clef étrangère pour \texttt{country.Code} &                   &                     &                    &                   \\
			\hline
		\end{tabular}
	}%
	\qquad
	\subfloat[Table \texttt{spoken}]{
		\begin{tabular}{|ccc|}
			\hline
			\texttt{Country}$ \star $                 & \texttt{Language}$ \star $ & \texttt{Percentage}                                      \\
			Clef étrangère pour \texttt{country.Code} &                            & Pourcentage de la population du pays qui parle la langue \\
			\hline
		\end{tabular}
	}%
	\qquad
	\subfloat[Table \texttt{economy}]{
		\begin{tabular}{|ccc|}
			\hline
			\texttt{Country}$ \star $                 & \texttt{GDP}                       & \texttt{Agriculture}                        \\
			Clef étrangère pour \texttt{country.Code} & Le PIB en million de dollars       & La part de l'agriculture dans le PIB, en \% \\
			\hline
			\texttt{Service}                          & \texttt{Industry}                  & \texttt{Inflation}                          \\
			La part des services dans le PIB          & La part de l'industrie dans le PIB & Le taux d'inflation                         \\
			\hline
			\texttt{Unemployment}                     &                                    &                                             \\
			Le taux de chômage                        &                                    &                                             \\
			\hline
		\end{tabular}
	}
	\caption{Structure de la base de données \textsc{Mondial}}
	\label{fig:mondiale}
\end{figure}


\begin{enonce}[Requêtes de base]
	Rédiger une requête SQL pour obtenir
	\ssques La liste des pays dont la population excède 60M habitants
	\ssques La même liste triée par ordre alphabétique
	\ssques La liste des pays et de leurs populations respectives, triée par ordre croissant de population
	\ssques Le nom des dix pays ayant la plus petite superficie
	\ssques Le nom des dix suivants
\end{enonce}

\begin{enonce}[Jointures]
	Rédiger une requête SQL pour obtenir
	\ssques Le nom des pays qui sont à cheval sur plusieurs continents
	\ssques Les pays du continent américain qui comptent moins de 10 habitants par $ \textrm{km}^2 $
	\ssques Les capitales européennes situées à une latitude supérieure à $ 60^{\circ} $
\end{enonce}

\begin{enonce}[Fonctions d'agrégation]
	\ssques Donner la liste ordonnée des dix langues parlées dans le plus de pays différents.
	\ssques Quelles sont les langues parlées dans exactement 6 pays ? Et de quel pays s'agit-il ?
	\ssques Quelles sont les langues parlées par moins de 30 000 personnes dans le monde ?
	\ssques Quelles sont les 5 langues les plus parlées en Afrique ? Préciser pour chacune d'elle le nombre de personnes qui la parlent.
\end{enonce}

\begin{enonce}[Sous-requêtes]
	\ssques Déterminer les pays majoritairement agricoles dont le taux de chômage est inférieur à la moyenne mondiale.
	\ssques Déterminer pour chaque continent le pays au taux d'inflation le plus faible parmi les pays majoritairement industriels.
\end{enonce}
