%! TEX root = ../main.tex

\titre{Requêtes SQL avancées}


Dans la suite, on dispose des tables décrites \autoref{fig:mondiale}. On marque par $ \star $ les colonnes qui forment la clef primaire de la table.
\begin{figure}[h!]
	\centering
	\subfloat[Table \texttt{country}]{
		\begin{tabular}{|cccccc|}
			\hline
			\texttt{Name} & \texttt{Code}$ \star $ & \texttt{Capital} & \texttt{Province} & \texttt{Area}                        & \texttt{Population} \\
			              &                        &                  &                   & Surface du pays en $ \textrm{km}^2 $ &                     \\
			\hline
		\end{tabular}
	}%
	\qquad
	\subfloat[Table \texttt{encompasses}]{
		\begin{tabular}{|ccc|}
			\hline
			\texttt{Country}$ \star $                 & \texttt{Continent}$ \star $ & \texttt{Percentage}                                 \\
			Clef étrangère pour \texttt{country.Code} &                             & Pourcentage de la surface du pays dans le continent \\
			\hline
		\end{tabular}
	}%
	\qquad
	\subfloat[Table \texttt{city}]{
		\begin{tabular}{|cccccc|}
			\hline
			\texttt{Name}$ \star $ & \texttt{Country}$ \star $                 & \texttt{Province} & \texttt{Population} & \texttt{Longitude} & \texttt{Latitude} \\
			                       & Clef étrangère pour \texttt{country.Code} &                   &                     &                    &                   \\
			\hline
		\end{tabular}
	}%
	\qquad
	\subfloat[Table \texttt{spoken}]{
		\begin{tabular}{|ccc|}
			\hline
			\texttt{Country}$ \star $                 & \texttt{Language}$ \star $ & \texttt{Percentage}                                      \\
			Clef étrangère pour \texttt{country.Code} &                            & Pourcentage de la population du pays qui parle la langue \\
			\hline
		\end{tabular}
	}%
	\qquad
	\subfloat[Table \texttt{economy}]{
		\begin{tabular}{|ccc|}
			\hline
			\texttt{Country}$ \star $                 & \texttt{GDP}                       & \texttt{Agriculture}                        \\
			Clef étrangère pour \texttt{country.Code} & Le PIB en million de dollars       & La part de l'agriculture dans le PIB, en \% \\
			\hline
			\texttt{Service}                          & \texttt{Industry}                  & \texttt{Inflation}                          \\
			La part des services dans le PIB          & La part de l'industrie dans le PIB & Le taux d'inflation                         \\
			\hline
			\texttt{Unemployment}                     &                                    &                                             \\
			Le taux de chômage                        &                                    &                                             \\
			\hline
		\end{tabular}
	}
	\caption{Structure de la base de données \textsc{Mondial}}
	\label{fig:mondiale}
\end{figure}

\section{Agrégations}

SQL permet de regrouper les résultats d'une requête grâce au mot clef \mintinline{sql}{GROUP BY}. Par exemple, on peut regrouper par continent dans la table \textsc{encompasses} en écrivant \mintinline{sql}{GROUP BY Continent}. Ensuite, les fonctions d'agrégation vues au TP précédent permettent de faire des calculs \textit{au sein chaque groupe}. Voici par exemple comment obtenir le nombre de pays de chaque continent 

\begin{minted}{sql}
    SELECT continent, COUNT(*)
        FROM encompasses
        GROUP BY continent;
\end{minted}

Ou encore, avec un exemple plus compliqué, voici comment obtenir la surface totale de chaque continent 

\begin{minted}{sql}
    SELECT e.continent, SUM(c.area * e.percentage)
        FROM encompasses as e JOIN country as c ON c.code = e.country
        GROUP BY e.continent;
\end{minted}

Attention, une fois qu'on regroupe selon une certaine colonne, on ne peut addresser les autres colonnes que par l'intermédiaire des fonctions d'aggrégation. Ainsi la requête suivante \textit{ne veut rien dire et n'est pas correcte} 

\begin{minted}{sql}
    SELECT e.continent, c.name -- NON on ne peut pas sélectionner directement c.name
        FROM encompasses as e JOIN country as c ON c.code = e.country
        GROUP BY e.continent;
\end{minted}

Ensuite, il est possible de filtrer les groupes avec le mot clef \mintinline{sql}{HAVING}. Par exemple, on peut ne sélectionner que les continents d'une surface plus grande que 10 000 000 $ \textrm{km}^2 $ avec 

\begin{minted}{sql}
    SELECT e.continent
        FROM encompasses as e JOIN country as c ON c.code = e.country
        GROUP BY e.continent
        HAVING sum(c.area * e.percentage) > 10000000;
\end{minted}

Notez la différence entre \mintinline{sql}{WHERE} et \mintinline{sql}{HAVING} : le premier filtre \textit{les lignes} avant de former les groupes, et le second filtre \textit{les groupes} après les avoir formés. La forme générale d'une requête est donc 

\begin{minted}{sql}
    SELECT column
        FROM table
        WHERE ligne_condition
        GROUP BY column
        HAVING groupe_condition
\end{minted}

\section{Sous-requêtes}

Le résultat d'une requête est lui-même une table qui peut être elle-même imbriquée dans une nouvelle requête. Par exemple, on peut récupérer la surface moyenne des pays avec la requête 

\begin{minted}{sql}
    SELECT avg(area) FROM country;
\end{minted}

En réutilisant la requête précédente on peut ainsi obtenir les pays dont la surface est plus grande que la moyenne : 

\begin{minted}{sql}
    SELECT name
        FROM country
        WHERE area > (SELECT avg(area) FROM country);
\end{minted}

Les requêtes imbriquées peuvent aussi être utilisées dans la clause \mintinline{sql}{FROM table} à la place de \mintinline{sql}{table}, mais dans cas il faut parfois donner un nom à la table et ses colonnes avec \mintinline{sql}{AS}. Par exemple, la requête suivante compte le nombre de langues parlées dans chaque pays 

\begin{minted}{sql}
    SELECT count(*) FROM spoken GROUP BY country;
\end{minted}

On réutilise cette requête pour obtenir le plus grand nombre de langues parlées dans un seul pays. Notez qu'on donne un nom à la colonne \mintinline{sql}{count(*)} pour pouvoir s'y référer dans le reste de la requête.

\begin{minted}{sql}
    SELECT MAX(n_language)
        FROM (SELECT count(*) as n_language FROM spoken GROUP BY country) as count_table;
\end{minted}

Le mot clef \mintinline{sql}{IN} peut être utilisé pour tester l'appartenance à une table (la table étant, en général, le résultat d'une requête imbriquée). Par exemple, on peut s'inspirer des requêtes ci-dessus pour obtenir le nombre de langues parlées dans les pays \textit{dont la surface est inférieure à la moyenne mondiale} 

\begin{minted}{sql}
    SELECT country, count(*)
        FROM spoken
        WHERE country IN (
            SELECT code
                FROM country
                WHERE area < (SELECT avg(area) FROM country)
        )
        GROUP BY country;
\end{minted}

Vous avez maintenant toutes les clefs pour finir les exercices suivants (les deux premiers sont des rappels du TP précédent). Si vous y arrivez, félicitations vous maîtrisez l'ensemble du programme de SQL de BCPST.

\begin{enonce}[Requêtes de base]
	Rédiger une requête SQL pour obtenir
	\ssques La liste des pays dont la population excède 60M habitants
	\ssques La même liste triée par ordre alphabétique
	\ssques La liste des pays et de leurs populations respectives, triée par ordre croissant de population
	\ssques Le nom des dix pays ayant la plus petite superficie
	\ssques Le nom des dix suivants
\end{enonce}

\begin{enonce}[Jointures]
	Rédiger une requête SQL pour obtenir
	\ssques Le nom des pays qui sont à cheval sur plusieurs continents
	\ssques Les pays du continent américain qui comptent moins de 10 habitants par $ \textrm{km}^2 $
	\ssques Les capitales européennes situées à une latitude supérieure à $ 60^{\circ} $
\end{enonce}

\begin{enonce}[Fonctions d'agrégation]
	\ssques Donner la liste ordonnée des dix langues parlées dans le plus de pays différents.
	\ssques Quelles sont les langues parlées dans exactement 6 pays ? Et de quel pays s'agit-il ?
	\ssques Quelles sont les langues parlées par moins de 30 000 personnes dans le monde ?
	\ssques Quelles sont les 5 langues les plus parlées en Afrique ? Préciser pour chacune d'elle le nombre de personnes qui la parlent.
\end{enonce}

\begin{enonce}[Sous-requêtes]
	\ssques Déterminer les pays majoritairement agricoles dont le taux de chômage est inférieur à la moyenne mondiale.
	\ssques Déterminer pour chaque continent le pays au taux d'inflation le plus faible parmi les pays majoritairement industriels.
    \ssques Déterminer les pays dans lesquels on parle le plus de langues.
    \ssques Quel est le continent le plus riche, en terme de PIB ?
\end{enonce}
