%! TEX root = ../main.tex

\titre{Introduction à la récursivité}

\commentaire{\textit{Qu'est-ce que la récursivité ?}\\
	Une fonction python peut appeler toute fonction définie (rappelez-vous d'exemples : on a utilisé \verb!abs! par exemple).\\
	En particulier, une fonction peut s'appeler \textit{elle-même} : on dit alors que c'est une \textit{fonction récursive}.
}

Un exemple classique est le calcul de factorielle $n$ : $n!$.
\begin{minted}{python}
def fact(n):
	if (n == 0):
		return 1
	else:
		return n * fact(n-1)
\end{minted}

\exo

\textit{Exponentiation rapide.} Testez à chaque fois vos fonctions !

\ques Ecrire une fonction \verb!puissance_for(x,n)! qui prend en entier un réel \verb!x! et un entier \verb!n!, et renvoie $x^n$, en utilisant une boucle \verb!for! pour calculer le résultat.

\ques Ecrire une fonction \verb!puissance_rec(x,n)! qui fait la même chose, mais en utilisant la récursivité (indice : $x^n = x*x^{n-1}$).

\ques Plus malin, plus rapide !\\
Ecrire une fonction \verb!puissance_rec_rapide(x,n)! qui fait la même chose, en étudiant la parité de \verb!n! pour appliquer des propriétés de la mise en puissance.

\exo

Le format de papier A0 correspond à un rectangle de largeur de 84,1cm et une longueur de 118,9cm. le format A1 est obtenu en coupant en deux parties égales le format A0, il a donc pour longueur la largeur de A0 et pour largeur la moitié de la longueur de A0. Sur le même principe une feuille A1 contient deux feuilles A2, une feuille A2 deux feuilles A3, etc...

Ecrire une fonction récursive \verb!format_papier(n)! prenant en paramètre un entier naturel $n$ et qui retourne longueur et largeur, dans cet ordre, d'une feuille de format An.

\exo

\textit{Suite de Fibonacci}\\
La suite de Fibonacci est définie par :
\begin{equation*}
	u_0 = 0, \\
	u_1 = 1, \\
	\forall n \in \mathds{N}, u_{n+2} = u_{n+1} + u_{n}
\end{equation*}

\ques Ecrire une fonction \verb!fibo(n)! qui renvoie le terme d'indice n de la suite de Fibonacci, de manière itérative (avec une boucle \verb!for!).

\ques Pareil, mais avec la récursivité. Est-ce efficace ? Comment faire mieux ?
